%! @brief code - format text as code.
%! @param 1 : text to format
%! @example :
%   \code{This text is formatted as code.} This text is not.
\newcommand{\code}[1]{\lstinline[style=InlineCode]!#1!}

%! @brief sourcecodeinline - add source code file from the code/ folder as paragraph
%! @param 1 : additional listings options, eg. caption="title"
%! @param 2 : file path
%! @example :
%   \sourcecodeinline{language=python,caption=script.py}{path/to/file/script.py}
\newcommand{\sourcecodeinline}[2]{
    \lstinputlisting[#1]{code/#2}
}

%! @brief sourcecode - add source code file from the code/ folder in text and toc
%! @param 1 : additional listings options, eg. caption="title"
%! @param 2 : file path
%! @param 3 : label for reference
%! @example :
%   \sourcecode{language=python,caption=source.c}{path/to/file/source.c}{sourcec}
\newcommand{\sourcecode}[3]{
    \phantomsection{}
    \captionof{figure}{Code source - #2}\label{#3}
    \sourcecodeinline{#1}{#2}
}

%! @brief simplefig - add figure in text
%! param 1 : caption
%! param 2 : content
%! param 3 : label
%! @example :
%   \simplefig{Legend of figure}{
%       \includegraphics[scale=0.5]{path/to/image/image.png}
%       }{fig:myimage}
\newcommand{\simplefig}[3]{
    \begin{figure}
        \caption{#1}\label{#3}
        \vspace{0.5cm}
        #2
    \end{figure}
}

%! @brief image - add image from the "images" folder in document and toc
%! @note : do not uses the figure env. as floats cannot be put in minipages.
%!      If you use the graybox environment, you should use this command to put
%!      figures and not the "figure" environment.
%! @param 1 : caption, used in toc
%! @param 2 : scale
%! @param 3 : image file path within the images/ folder
%! @param 4 : label
%! @example :
%   \image{Legend of image}{0.5}{image.png}{fig:myimage}
\newcommand{\image}[4]{
    \phantomsection{}
    \begin{center}
        \captionof{figure}{#1}\label{#4}
        \vspace{0.5cm}
        \includegraphics[scale=#2]{images/#3}
    \end{center}
}

%! @brief screenshot - add screenshots from the images folder in document and toc
%! @details : This command will add the capture in its own page in landscape for
%!     at for better readability.
%! @warning : Do NOT use in minipages, including the graybox environment ! 
%! @param 1 : caption, used in toc
%! @param 2 : scale (default 0.57)
%! @param 3 : image file name in the captures folder
%! @param 4 : label
%! @example :
%   \screenshot{Legend of screenshot}{}{screenshot.png}{fig:myscreenshot}
%   \screenshot{Legend of screenshot}{0.8}{screenshot.png}{fig:myscreenshot}
\newcommand{\screenshot}[4]{
    \begin{landscape}
        \begin{figure}
            \centering
            \caption{Capture d'écran - #1}\label{#4}
            \vspace{0.5cm}
            \ifthenelse{\equal{#2}{}}{
                \includegraphics[scale=0.57]{images/#3}
            }{
                \includegraphics[scale=#2]{images/#3}
            }
        \end{figure}
    \end{landscape}
}

%! @brief screenshotsm - add smartphone screenshots in document and toc
%! @details : This command will add up to two capture side by side in the
%!     document. This is done for portait captures.
%! @warning : Do NOT use in minipages, including the graybox environment ! 
%! @param 1 : caption, used in toc
%! @param 2 : scale (default to 0.2)
%! @param 3 : image file name in the captures folder
%! @param 4 : (optional) image file name in the captures folder
%! @param 5 : label for both
%! @example :
%   \screenshotsm{Legend of screenshot}{}{screenshot1.png}{}{fig:myscreenshot}
%   \screenshotsm{Legend of screenshot}{0.8}{screenshot1.png}{screenshot2.png}{fig:myscreenshots}
\newcommand{\screenshotsm}[5]{
    \begin{figure}
        \caption{Capture d'écran - #1}\label{#5}
        \vspace{0.5cm}
        \ifthenelse{\equal{#2}{}}{
            \includegraphics[scale=0.2]{images/#3}
        }{
            \includegraphics[scale=#2]{images/#3}
        }
        \ifthenelse{\equal{#4}{}}{
        }{
            \ifthenelse{\equal{#2}{}}{
                \includegraphics[scale=0.2]{images/#4}
            }{
                \includegraphics[scale=#2]{images/#4}
            }
        }
    \end{figure}
}

%! @brief starchapter - add starred chapter in both text and toc
%! @param 1 : the chapter name
%! @example :
%   \starchapter{My chapter name}
\newcommand{\starchapter}[1]{
    \phantomsection{}
    \chapter*{#1}
    \addcontentsline{toc}{chapter}{#1}
}

%! @brief starsection - add starred section in both text and toc
%! @param 1 : the section name
%! @example :
%   \starsection{My section name}
\newcommand{\starsection}[1]{
    \phantomsection{}
    \section*{#1}
    \addcontentsline{toc}{section}{#1}
}

%! @brief starsubsection - add starred subsection in both text and toc
%! @param 1 : the subsection name
%! @example :
%   \starsubsection{My subsection name}
\newcommand{\starsubsection}[1]{
    \phantomsection{}
    \subsection*{#1}
    \addcontentsline{toc}{subsection}{#1}
}

%! @brief addbookmark - add a bookmark
%! @source : https://tex.stackexchange.com/a/434026
%! @param 1 : type (toc, titlepage, etc)
%! @param 2 : level (chapter, section, etc)
%! @param 3 : text to appear
%! @example :
%   \addbookmark{toc}{section}{My section name}
\newcommand{\addbookmark}[3]{
    \hypertarget{#1}{}
    \bookmark[dest=#1, level=#2]{#3}
}
