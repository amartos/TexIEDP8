\title{Matière du devoir} % Matière

\newcommand{\name}{John Doe} % Prénom Nom
\newcommand{\studentid}{123456789} % n° étudiant
\newcommand{\subtitle}{Titre du devoir} % Titre du devoir

% la commande section donne maintenant: 
% \section{À rendre} => Exercice 1.2 À rendre
% à commenter pour remettre les titres de sections habituels
\titleformat{\section}{\bfseries \large}{Exercice \thesection}{1em}{}

% Définition du style principal des codes sources
% styles définis dans src/codeblocks.tex
\lstdefinestyle{CodeBox}{
    style=tab4, % la largeur de tabulation vaut 4 espaces
    style=linesnb5, % numéro de lignes, toutes les 5 lignes
    style=box, % enferme les codes sources dans des boites colorées
    style=fnttstyle, % police type typewriter
    style=syntaxcol, % coloration syntaxique
}

% Définition du style des bouts de code insérés dans le texte.
% styles définis dans src/codeblocks.tex
\lstdefinestyle{InlineCode}{
    style=ttstyle
}
