\chapter{Titre du chapitre}

\section{À rendre}

Consignes du devoir.

\begin{graybox}
    \textbf{Réponse:}

    \subsection{Première sous-section}

    Texte de réponse.

    \sourcecodeinline{language=python,caption=script.py}{script.py}
    \sourcecode{language=c,caption=source.c}{script.c}{scriptcinline}

    \subsection{Deuxième sous-section}

    Texte de réponse. Référence code \vpageref{scriptc}.

    Sources des images: \citeurl{dranville_english_2019,
    softwarelibrivox_mozilla_foundation_and_contributorsscreenshotvulphere_english_2020}

    \newpage % évite la séparation légende - image
    \image{Légende de l'image}{0.3}{Firefox_Developer_Edition_75_0b9_LibriVox.png}{fig:myimage}

    \subsubsection{Sous-section}

    Texte de réponse.

    \subsection{Conlusion}

    La conclusion.

\end{graybox}

\newpage

\section{Codes sources}

\sourcecode{language=python,caption=script.py}{script.py}{scripty}
\sourcecode{language=c,caption=source.c}{script.c}{scriptc}

\subsection{Captures d'écrans}

\screenshot{Légende de la capture}{}{Firefox_Developer_Edition_75_0b9_LibriVox.png}{fig:screenshot}
\screenshot{Légende de la capture}{0.2}{Firefox_Developer_Edition_75_0b9_LibriVox.png}{fig:screenshotreduced}

% Source de l'image : https://commons.wikimedia.org/wiki/File:Screenshot_Wikipedia_Mobile_Smartphone.png
\screenshotsm{Légende de la capture}{}{Screenshot_Wikipedia_Mobile_Smartphone.png}{}{fig:mysinglescreenshot}
\screenshotsm{Légende de la capture}{0.2}
    {Screenshot_Wikipedia_Mobile_Smartphone.png}%
    {Screenshot_Wikipedia_Mobile_Smartphone.png}%
{fig:mydoublescreenshots}
